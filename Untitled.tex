\documentclass{article}

\usepackage{arxiv}

\usepackage[utf8]{inputenc} % allow utf-8 input
\usepackage[T1]{fontenc}    % use 8-bit T1 fonts
\usepackage{lmodern}        % https://github.com/rstudio/rticles/issues/343
\usepackage{hyperref}       % hyperlinks
\usepackage{url}            % simple URL typesetting
\usepackage{booktabs}       % professional-quality tables
\usepackage{amsfonts}       % blackboard math symbols
\usepackage{nicefrac}       % compact symbols for 1/2, etc.
\usepackage{microtype}      % microtypography
\usepackage{graphicx}

\title{An analysis using simulation to compare several moving average
techniques for time series data.}

\author{
    Bishal sarker
   \\
    Applied Statistics \\
    University of Dhaka \\
  Dhaka, Bangladesh \\
  \texttt{\href{mailto:sarkerbishal02@gmail.com}{\nolinkurl{sarkerbishal02@gmail.com}}} \\
   \And
    Elias D. Striatum
   \\
    Department of Electrical Engineering \\
    Mount-Sheikh University \\
  Santa Narimana, Levand \\
  \texttt{\href{mailto:stariate@ee.mount-sheikh.edu}{\nolinkurl{stariate@ee.mount-sheikh.edu}}} \\
  }


% tightlist command for lists without linebreak
\providecommand{\tightlist}{%
  \setlength{\itemsep}{0pt}\setlength{\parskip}{0pt}}


% Pandoc citation processing
\newlength{\cslhangindent}
\setlength{\cslhangindent}{1.5em}
\newlength{\csllabelwidth}
\setlength{\csllabelwidth}{3em}
\newlength{\cslentryspacingunit} % times entry-spacing
\setlength{\cslentryspacingunit}{\parskip}
% for Pandoc 2.8 to 2.10.1
\newenvironment{cslreferences}%
  {}%
  {\par}
% For Pandoc 2.11+
\newenvironment{CSLReferences}[2] % #1 hanging-ident, #2 entry spacing
 {% don't indent paragraphs
  \setlength{\parindent}{0pt}
  % turn on hanging indent if param 1 is 1
  \ifodd #1
  \let\oldpar\par
  \def\par{\hangindent=\cslhangindent\oldpar}
  \fi
  % set entry spacing
  \setlength{\parskip}{#2\cslentryspacingunit}
 }%
 {}
\usepackage{calc}
\newcommand{\CSLBlock}[1]{#1\hfill\break}
\newcommand{\CSLLeftMargin}[1]{\parbox[t]{\csllabelwidth}{#1}}
\newcommand{\CSLRightInline}[1]{\parbox[t]{\linewidth - \csllabelwidth}{#1}\break}
\newcommand{\CSLIndent}[1]{\hspace{\cslhangindent}#1}

\usepackage{booktabs}
\usepackage{longtable}
\usepackage{array}
\usepackage{multirow}
\usepackage{wrapfig}
\usepackage{float}
\usepackage{colortbl}
\usepackage{pdflscape}
\usepackage{tabu}
\usepackage{threeparttable}
\usepackage{threeparttablex}
\usepackage[normalem]{ulem}
\usepackage{makecell}
\usepackage{xcolor}
\begin{document}
\maketitle


\begin{abstract}
Enter the text of your abstract here.
\end{abstract}

\keywords{
    blah
   \and
    blee
   \and
    bloo
   \and
    these are optional and can be removed
  }

\hypertarget{introduction}{%
\section{Introduction:}\label{introduction}}

\hypertarget{background-of-the-study}{%
\subsection{Background of The Study}\label{background-of-the-study}}

Data for time series is gathered from various points throughout time. As
a result, the data set has a great deal of variety. So, a technique
known as smoothing is employed to lessen these variations. Techniques
for removing noise from a time series of data include smoothing
techniques. It aids in determining the datset's trend. When data is
compiled, any volatility or other types of noise can be removed or
reduced. Data smoothing is the term for this. Data smoothing is based on
the notion that it can recognize simpler changes to assist in the
prediction of various trends and patterns. It serves as a tool for
statisticians or traders who must examine a lot of data, which is
frequently challenging.

\hypertarget{objective-of-the-study}{%
\subsection{Objective of The Study}\label{objective-of-the-study}}

Among all others smoothing methods moving average methods is the oldest
and simplest smoothing methods. The main object of that study is
comparing different types of moving average smoothing techniques such as
simple moving average(SMA),exponentially-weighted moving average (EWMA),
weighted moving average (WMA), double exponential moving
average(DEMA),Hull moving average(HMA), Zero lag exponential moving
average(ZLEMA) etc. Among them the syudy compares SMA, EWMA, WMA,
DEMA,HMA and ZLEMA.

\hypertarget{literature-review}{%
\section{Literature review}\label{literature-review}}

Raudys, Lenčiauskas, and Malčius (2013) smoothed financial data using
the moving average.

Ivanovski, Milenkovski, and Narasanov (2018) extrapolated the number of
tourists using the moving average. The study aids in making wise
decisions for the future.

Hameed (2015) compares every smoothing method currently in use to
forecast future demand for private universities in Bangladesh. They
contrast different kinds of currently used smoothing techniques and
discover that Holt's method provides the optimum accuracy for their
work.

For the purpose of early detection of infectious disease outbreaks, Yang
et al. (2018) conducts simulation-based studies on the comparison of
statistical and time series forecasting techniques. Here, various
approaches are discussed in an effort to use simulation to produce the
greatest results.

Sinaga and Irawati (2020) studied about medical disposable supply demand
forecasting by moving average and exponential moving average method.

For predicting power load, Karim and Alwi (2013) employed exponential
smoothing and moving average, and exponential moving average
outperformed moving average.

Fong et al. (2020) tried to find an accurate early forecasting model
from Small dataset of 2019-nCoV Novel Coronavirus outbreak.

\hypertarget{methodology}{%
\section{Methodology}\label{methodology}}

\hypertarget{simple-moving-average-sma}{%
\subsection{Simple moving average
(SMA)}\label{simple-moving-average-sma}}

By averaging several different subsets of the entire data set, a simple
moving average (SMA) statistical technique is utilized to examine data
points. Because it is calculated by averaging a predetermined number of
successive data points with equal weight for each, it is referred to as
being ``simple.'' The SMA is used to highlight long-term trends or
patterns in the data and to smooth out short-term volatility in the
data.The formula for SMA is given below,

\[
{T_t = \frac{\sum_{i=-m}^{m} Y_{t+i}}{k}} 
\]

here k is the number of order and \(Y_i\) is the observation.
m=\(\frac{k-1}{2}\) is the half width of a moving average as the number
of points. For example, a arithmetic moving average of 3 ordered at time
t is \(T_t=\frac{Y_{t-1}+Y_t+Y_{t+1}}{3}\).

Simple Moving Average (SMA) and Exponential Moving Average (EMA) both
measure trend direction over time. SMA only determines an average, but
EMA gives more weight to data that is more recent.

\hypertarget{exponentially-moving-average-ema}{%
\subsection{Exponentially moving average
(EMA)}\label{exponentially-moving-average-ema}}

Simple Moving Average (SMA) and Exponential Moving Average (EMA) both
measure trend direction over time. SMA only determines an average, but
EMA gives more weight to data that is more recent. \[
EMA = C – P  \frac{2}{(n+1)} + P
\]

where C and P are current data point and an exponential moving average
of the previous period (simple average used for the first period)
respectively. THe formula is given below,

\hypertarget{weighted-moving-average}{%
\subsection{Weighted moving average}\label{weighted-moving-average}}

When calculating the weighted moving average, recent data points are
given more weighting than historical data points. When added together,
the weights' total value ought to be 100\%, or 1. The weighting factor
used to calculate the WMA is determined by the period selected for the
indicator. For example, a 5 period WMA would be calculated as follows:

\[
WMA = \frac{(5P_{1}  + 4P_2  + 3P_3  + 2P_4  + 1P_5 )} {(5 + 4+ 3 + 2 + 1)}
\]

Where, \(P_1\)= current price \(P_2\) = price one bar ago and so on.

\hypertarget{zlema}{%
\subsection{ZLEMA}\label{zlema}}

John Ehlers and Ric Way created the Zero Lag Exponential Moving Average
(ZLEMA) indicator. The goal is to get rid of the inherent lag that all
averages and other trend following indicators have.

This is what the ZLEMA aims to accomplish by tracking recent prices more
closely than historical prices, much like a standard EMA but with an
even greater emphasis on recent prices. A moving average with less lag
and good smoothing is the end product.

The benefits of using zero lag moving averages are as follows:

\begin{itemize}
\tightlist
\item
  A moving average exponential without lag that is more responsive to
  current price changes.
\item
  The indicator is applicable across all instruments and timeframes.
\item
  The indicator may be used as a signal or as a filter for signals.
\end{itemize}

Formula are given below ,

\[
\alpha = \frac{2}{n+1}  
\] \[
Z_{t} = (1 - \alpha)Z_t + \alpha (T_t - T_{t-\frac{n-1}{2}})
\]

Where , n is the number of period. \(\alpha\) represents the lag and
\(Z_t\) represents the ZLEMA moving average at t time points.

\hypertarget{dema}{%
\subsection{DEMA}\label{dema}}

A level component and a trend component are used in double exponential
smoothing at each period. Two weights, also known as smoothing
parameters, are used in double exponential smoothing to update the
components at each time.The formula is given below,

\[
L_{t} = \alpha Y_{t} + (1-\alpha)(L_{t-1} + T_{t-1})
\] \[
T_t = \gamma (L_t – L_{t –1}) + (1 – \gamma) T_{t –1} 
\] \[
\hat Y= L_{t –1} + T_{t –1}
\] Here , \(L_t\) level at time t.\\
\(\alpha\) weight for the level.\\
\(T_t\) trend at time t\\
\(\gamma\) weight for the trend\\
\(Y_t\) data value at time t\\
\(\hat Y_t\) fitted value, or one-step-ahead forecast, at time t\\

\hypertarget{analysis}{%
\section{Analysis}\label{analysis}}

\begin{table}[H]

\caption{\label{tab:unnamed-chunk-10}Smoothing comparison table}
\centering
\begin{tabular}[t]{lrrlrrlrr}
\toprule
\multicolumn{3}{c}{Normal Distribution} & \multicolumn{3}{c}{Poisson Distribution} & \multicolumn{3}{c}{Weibull Distribution} \\
\cmidrule(l{3pt}r{3pt}){1-3} \cmidrule(l{3pt}r{3pt}){4-6} \cmidrule(l{3pt}r{3pt}){7-9}
TYPE & MAE & MSE & TYPE & MAE & MSE & TYPE & MAE & MSE\\
\midrule
SMA & 2.813 & 12.440 & SMA & 1.770 & 4.978 & SMA & 0.319 & 0.164\\
EMA & 1.624 & 4.142 & EMA & 1.028 & 1.660 & EMA & 0.184 & 0.055\\
DEMA & 0.742 & 0.864 & DEMA & 0.469 & 0.346 & DEMA & 0.084 & 0.011\\
WMA & 1.875 & 5.529 & WMA & 1.180 & 2.212 & WMA & 0.213 & 0.073\\
HMA & 1.875 & 5.529 & HMA & 1.180 & 2.212 & HMA & 0.213 & 0.073\\
\addlinespace
ZLEMA & 1.873 & 5.506 & ZLEMA & 1.187 & 2.216 & ZLEMA & 0.213 & 0.073\\
\bottomrule
\end{tabular}
\end{table}

\begin{table}[H]

\caption{\label{tab:unnamed-chunk-11}Smoothing comparison table}
\centering
\begin{tabular}[t]{lrrlrrlrr}
\toprule
\multicolumn{3}{c}{Gamma Distribution} & \multicolumn{3}{c}{T Distribution} & \multicolumn{3}{c}{Exponential Distribution} \\
\cmidrule(l{3pt}r{3pt}){1-3} \cmidrule(l{3pt}r{3pt}){4-6} \cmidrule(l{3pt}r{3pt}){7-9}
TYPE & MAE & MSE & TYPE & MAE & MSE & TYPE & MAE & MSE\\
\midrule
SMA & 1.770 & 4.978 & SMA & 0.622 & 0.629 & SMA & 0.100 & 0.020\\
EMA & 1.028 & 1.660 & EMA & 0.359 & 0.209 & EMA & 0.059 & 0.007\\
DEMA & 0.469 & 0.346 & DEMA & 0.164 & 0.044 & DEMA & 0.027 & 0.001\\
WMA & 1.180 & 2.212 & WMA & 0.415 & 0.279 & WMA & 0.066 & 0.009\\
HMA & 1.180 & 2.212 & HMA & 0.415 & 0.279 & HMA & 0.066 & 0.009\\
\addlinespace
ZLEMA & 1.187 & 2.216 & ZLEMA & 0.414 & 0.279 & ZLEMA & 0.068 & 0.009\\
\bottomrule
\end{tabular}
\end{table}

\hypertarget{refs}{}
\begin{CSLReferences}{1}{0}
\leavevmode\vadjust pre{\hypertarget{ref-fong2020finding}{}}%
Fong, Simon James, Gloria Li, Nilanjan Dey, Rubén González Crespo, and
Enrique Herrera-Viedma. 2020. {``Finding an Accurate Early Forecasting
Model from Small Dataset: A Case of 2019-Ncov Novel Coronavirus
Outbreak.''} \emph{arXiv Preprint arXiv:2003.10776}.

\leavevmode\vadjust pre{\hypertarget{ref-hameed2015smoothing}{}}%
Hameed, Haifaa Hussein. 2015. {``Smoothing Techniques for Time Series
Forecasting.''} Master's thesis, Eastern Mediterranean University
(EMU)-Do{ğ}u Akdeniz {Ü}niversitesi (DA{Ü}).

\leavevmode\vadjust pre{\hypertarget{ref-ivanovski2018time}{}}%
Ivanovski, Zoran, Ace Milenkovski, and Zoran Narasanov. 2018. {``TIME
SERIES FORECASTING USING a MOVING AVERAGE MODEL FOR EXTRAPOLATION OF
NUMBER OF TOURIST.''} \emph{UTMS Journal of Economics} 9 (2).

\leavevmode\vadjust pre{\hypertarget{ref-karim2013electricity}{}}%
Karim, SA Abdul, and Saiful Azli Alwi. 2013. {``Electricity Load
Forecasting in UTP Using Moving Averages and Exponential Smoothing
Techniques.''} \emph{Applied Mathematical Sciences} 7 (77-80): 4003--14.

\leavevmode\vadjust pre{\hypertarget{ref-raudys2013moving}{}}%
Raudys, Aistis, Vaidotas Lenčiauskas, and Edmundas Malčius. 2013.
{``Moving Averages for Financial Data Smoothing.''} In \emph{Information
and Software Technologies: 19th International Conference, ICIST 2013,
Kaunas, Lithuania, October 2013. Proceedings 19}, 34--45. Springer.

\leavevmode\vadjust pre{\hypertarget{ref-sinaga2020medical}{}}%
Sinaga, H, and N Irawati. 2020. {``A Medical Disposable Supply Demand
Forecasting by Moving Average and Exponential Smoothing Method.''} In
\emph{Proceedings of the 2nd Workshop on Multidisciplinary and
Applications (WMA) 2018, 24-25 January 2018, Padang, Indonesia}.

\leavevmode\vadjust pre{\hypertarget{ref-yang2018simulation}{}}%
Yang, Eunjoo, Hyun Woo Park, Yeon Hwa Choi, Jusim Kim, Lkhagvadorj
Munkhdalai, Ibrahim Musa, and Keun Ho Ryu. 2018. {``A Simulation-Based
Study on the Comparison of Statistical and Time Series Forecasting
Methods for Early Detection of Infectious Disease Outbreaks.''}
\emph{International Journal of Environmental Research and Public Health}
15 (5): 966.

\end{CSLReferences}

\bibliographystyle{unsrt}
\bibliography{references.bib}


\end{document}
